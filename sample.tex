\documentclass[12pt,a4paper,violet]{bbe}
\usepackage{blindtext}
\usepackage{listings}
\usepackage{xcolor}
\begin{document}
	
	%New colors defined below
\definecolor{codegreen}{rgb}{0,0.6,0}
\definecolor{codegray}{rgb}{0.5,0.5,0.5}
\definecolor{codepurple}{rgb}{0.58,0,0.82}
\definecolor{codebackcolour}{rgb}{0.95,0.95,0.92}

  	
    %added be me.
    %Code listing style named "mystyle"
    \lstdefinestyle{mystyle}{
      backgroundcolor=\color{codebackcolour},
      commentstyle=\color{codegreen},
      keywordstyle=\color{magenta},
      numberstyle=\tiny\color{codegray},
      stringstyle=\color{codepurple},
      basicstyle=\ttfamily\footnotesize,
      breakatwhitespace=false,         
      breaklines=true,                 
      captionpos=b,                    
      keepspaces=true,                 
      numbers=left,                    
      numbersep=5pt,                  
      showspaces=false,                
      showstringspaces=false,
      showtabs=false,                  
      tabsize=2
    }




	\chapter{Setting up}
	\section{Variables}
	This is where we start
	
	
    
  
    
    
    %"mystyle" code listing set
    \lstset{style=mystyle}
   

	
	
	\lstinputlisting[language=Python]{python/watermellon.py}
	
	
	
	\begin{figure}[h]
        \centering
        \includegraphics[width=0.25\textwidth]{icons/python.png}
        \caption{A python Logo}
        \label{fig:mesh1}
    \end{figure}
    
    \begin{lstlisting}
    print("Hello World")\end{lstlisting}
	
	\begin{definition}
	You can always use an online editor
	\end{definition}
	
	\begin{theorem}
	You can always use an online editor
	\end{theorem}
	
	\begin{solution}
	You can always use an online editor
	\end{solution}
	
	\begin{activity}
	This is an activity
	\end{activity}
	
	\begin{example}
	this is an example
	\end{example}
	
	
	\includegraphics[scale=0.6]{icons/python.png}
	 
	
	\begin{property}
	    this is a property
	\end{property}

    \begin{hint}
        this is a hint
    \end{hint}
	
	\begin{remark}
	This is remark this is me
	\end{remark}
	
	\begin{mynote}
	this is a simple remark
	\end{mynote}
	
    
    
	\chapter{Setting up}
	\section{Variables}
	
	
	
	\blinddocument
\end{document}